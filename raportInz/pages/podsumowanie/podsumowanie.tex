\section{Podsumowanie projektu}

\textbf{Implementacja algorytmu i symulatora}

Analizując wyniki otrzymane podczas testowania algorytmu możemy zauważyć że algorytm poprawnie wyznacza 
najkrótszą możliwą ścieżkę. Podczas testów sprawdzono wpływ różnych funkcji heurestycznych na otrzymaną ścieżkę.
Na ich podstawie widać że takie funkcje mają duże znaczenie i w zależności od skomplikowania
otoczenia, odległości oraz dostępnych ruchów(możliwość przejścia po przekątnej) otrzymujemy ścieżki o różnym poziomie skomplikowania.
Zaimplementowanie algorytmu w pythonie pozwoliło na szybkie i sprawne jego przetestowanie, jednak w przypadku dużych i skomplikowanych map 
można było odczuć coraz dłuższe przeliczanie. W znacznej mierze wynika to ze ścisłego połączenia algorytmu z symulatorem i co za tym idzie wymuszone 
ciągłe przeszukiwanie tablicy z przeszkodami. Można to naprawić poprzez wygenerowanie grafu z tylko
najpotrzebniejszymi informacjami przed uruchomieniem algorytmu, dzięki czemu zredukuje się liczba niepotrzebnych wywołań funkcji.

Równocześnie została potwierdzona teza z literatury, mówiąca o prostocie algorytmu a co za tym idzie możliwości
stosowania go w różnych zastosowaniach między innymi do wyznaczania ścieżki w sieciach komputerowych co umożliwia 
implementacje komunikacji na mikrokontrolerach pracujących w rozwiązaniach IOT. 

\textbf{Budowa robota}

Zbudowany robot był zbyt lekki na przodzie przez co w niektórych przypadkach (szczególnie podczas ruszania) podnosił się. Komunikacja bazująca na 
wifi i serwerze tcp sprawdzała się dobrze i robot odbierał polecenia bez zauważalnych opóźnień. Dzięki systemowi poleceń tekstowych możliwe jest 
proste rozbudowanie oprogramowania pozwalające na odbieranie i prezentowanie danych takich jak prędkość kół napędowych. Ostatecznie nie udało się przetestować 
działania algorytmu na robocie, ponieważ uszkodzeniu uległ główny sterownik. Prawdopodobnie uszkodzenie wynika z zastosowania nieoryginalnego 
sterownika do silników i pojawienia się na liniach sygnałowych napięcia większego niż 3,3V. 

Należy pamiętać że w prawdziwym autonomicznym robocie (na przykład przewożącym towar w magazynie) sam algorytm wyznaczający trasę nie będzie wystarczający. 
Algorytm bazuje na mapie otoczenia a w praktyce ta będzie się bardzo często zmieniać więc najlepiej aby robot sam skanował otoczenie
i dynamicznie dobierał optymalne rozwiązanie. Finalnie duży wpływ na czas przejazdu będzie miała heurestyka, krótsza trasa z dużą ilością skrętów może mieć 
większy czas przejazdu niż potencjalnie dłuższa, na której robot może osiągnąć prędkość maksymalną. 