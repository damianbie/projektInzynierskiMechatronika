\section{Wprowadzenie}
Od lat można zauważyć zwiększająca się popularność różnych robotów, które bazując na wprowadzonej do systemu mapie
autonomicznie poruszają się po terenie tak aby osiągnąć określony cel.
Celem projektu inżynierskiego będzie zaimplementowanie algorytmu wyszukującym najkrótszą ścieżkę.
Do przetestowania algorytmu zostanie napisany prosty program symulujący robota poruszającego się po wyznaczonej ścieżce.
Kolejnym etap projektu to zbudowanie robota mającego zweryfikować zaproponowaną przez algorytm ścieżkę. 

Głównym powodem podjęcia się implementacji takiego algorytmu jest chęć zapoznania się z algorytmem planującym ścieżke 
i próba zaimplementowania takiego rozwiązania na fizycznym robocie.

% Potencjalne zastosowania autonomicznych są 

\textbf{Nazakres pracy składa się:}
\begin{itemize}
	\item Przegląd literatury
	\item Implementacja algorytmu A* w języku Python
	\item Symulacja działania pracy algorytmu
	\item Weryfikacja na obiekcie rzeczywistym
\end{itemize}

\textbf{Omówienie rozdziałów}

Rozdział pierwszy - zawiera przegląd literatury oraz istniejących rozwiązań. 
Na ich podstawie określona zostanie ogólna struktura własnego rozwiązania.

Rozdział drugi - skupia się na genezie, charakterystyce i opisie działania wybranego algorytmu wyszukiwania najkrótszej ścieżki.
Dodatkowo zostaną przedstawione inne algorytmy zwiększającą autonomie robota mobilnego.

Rozdział trzeci - w tym rozdziale zostanie przedstawione napisany algorytm oraz środowisko symulacyjne w języku Python.

Rozdział czwarty - omawia budowę robota mobilnego oraz jego program sterujący.
W szczególności skupiono się na segment komunikacji oraz regulatorze PID sterującym silnikami.

Rozdział piąty - przedstawia przeprowadzone testy wykonane w środowisku symulacyjnym oraz rzeczywistym. 