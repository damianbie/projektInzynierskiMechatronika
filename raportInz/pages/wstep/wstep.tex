\section{Wprowadzenie}
Od lat można zauważyć zwiększająca się popularność różnych robotów, które bazując na wprowadzonej do systemu mapie
autonomicznie poruszają się po terenie tak aby osiągnąć określony cel.
Celem projektu inżynierskiego będzie zaimplementowanie algorytmu wyszukującym najkrótszą ścieżkę.
Do przetestowania algorytmu zostanie napisany prosty program symulujący robota poruszającego się po wyznaczonej ścieżce.
Kolejnym etap projektu to zbudowanie robota mającego zweryfikować zaproponowaną przez algorytm ścieżkę. 

Głównym powodem podjęcia się implementacji takiego algorytmu jest chęć zapoznania się z systemami planującymi ruch robota
i próba zaimplementowania takiego rozwiązania na fizycznym robocie.

% Potencjalne zastosowania autonomicznych są 
\replaced[id=PP]{
	Ze względu na cel projektu inżynierskiego oraz powód jego realizacji, przyjęto, że praca będzie złożona z przeglądu literatury, implementacji i symulacji algorytmu A* w języku Python oraz weryfikacji przyjętych rozwiązań obiekcie rzeczywistym.
}{\textbf{Na zakres pracy składa się: ..... }
%\begin{itemize}
%	\item Przegląd literatury
%	\item Implementacja algorytmu A* w języku Python
%	\item Symulacja działania pracy algorytmu
%	\item Weryfikacja na obiekcie rzeczywistym
%\end{itemize}
}

\replaced{Mając na uwadze zakres pracy, jej treść podzielono na szereg rozdziałów. 
Rozdział pierwszy stanowi przegląd literatury oraz istniejących rozwiązań, które będą podstawą założeń przyjętych we opracowywanym rozwiązaniu. Rozdział drugi skupia się na genezie, charakterystyce i opisie działania wybranego algorytmu wyszukiwania najkrótszej ścieżki. Dodatkowo zostaną przedstawione inne algorytmy zwiększającą autonomie robota mobilnego. Rozdział trzeci przedstawia własną implementacje algorytmu A* oraz środowisko symulacyjne w języku Python. Kolejny rozdział to omówienie budowy robota mobilnego oraz jego programu sterującego. W tym kontekście szczególną uwagę poświęcono komunikacji oraz sterowaniu silnikami Rozdział piąty stanowi podsumowanie pracy i przedstawia przeprowadzone testy wykonane w środowisku symulacyjnym oraz rzeczywistym. }{
\textbf{Omówienie rozdziałów}

Rozdział pierwszy - zawiera przegląd literatury oraz istniejących rozwiązań. 
Na ich podstawie określona zostanie ogólna struktura własnego rozwiązania.

Rozdział drugi - skupia się na genezie, charakterystyce i opisie działania wybranego algorytmu wyszukiwania najkrótszej ścieżki.
Dodatkowo zostaną przedstawione inne algorytmy zwiększającą autonomie robota mobilnego.

Rozdział trzeci - w tym rozdziale zostanie przedstawiony napisany algorytm oraz środowisko symulacyjne w języku Python.

Rozdział czwarty - omawia budowę robota mobilnego oraz jego program sterujący.
W szczególności skupiono się na komunikacji oraz sterowaniu silnikami.

Rozdział piąty - przedstawia przeprowadzone testy wykonane w środowisku symulacyjnym oraz rzeczywistym. }