\subsection{Opis działania algorytmu}

A* to heurestyczny algorytm wyznaczający najkrótszą możliwą ścieżke w grafie. 
Jest to algorytm zupełny i optymalny a więc zawsze zostanie wyznaczona optymalne 
rozwiązanie. Ze względu na przeszukiwanie oparte na grafie algorytm najlepiej działa na strukturze drzewiastej.
Zadaniem algorytmu jest minimalizacja funkcji:
\begin{equation}
	f(x)=h(x) + g(x)
	\label{Eq:funkcjaKosztuAStar}
\end{equation}
gdzie: $f(x)$ - minimalizowana funkcja, $g(x)$ - to rzeczywisty koszt dojścia do punktu x.

Funckja $h(x)$ to funkcja heurestyczna oszacowująca ona koszt dotarcia od punktu x do wierzchołka docelowego

Zalety:
\begin{itemize}
    \item jest kompletny i optymalny
    \item może przeszukować skomplikowane mapy
    \item jest najwydajnieszym takim algorytmem
\end{itemize}
Wady:
\begin{itemize}
    \item jego wydajność w znacznej mierze zależy od funkcji heurestycznej
    \item Każda akcja ma stały koszt wykonania
    \item nie nadaje się do często zmieniającego się otoczenia robota, wymaga ponownego przeliczenia
    
\end{itemize}


\textbf{Przykładowe funkcje heurestyczne:}
\begin{itemize}
    \item Funkcja euklidesowa
    \begin{equation}
        h(x)= 10 * \sqrt[2]{(x_1 - x_2)^2 + (y_1 - y_2)^2}
        \label{Eq:heuresticEucalides}
    \end{equation}
    \item Geometria Manhattanu (innaczej metryka miejska)
    \begin{equation}
        h(x)= |x_2 - x_1| + |y_2 - y_1|
        \label{Eq:heuresticManhattanu}
    \end{equation}
\end{itemize}
Gdzie: $x_1$ i $y_1$ to współrzędne wyznaczanego punktu, $x_2$ i $y_2$ to koordynaty celu