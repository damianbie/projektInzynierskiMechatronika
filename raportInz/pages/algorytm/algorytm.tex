\section{Algorytm A*}
\subsection{Geneza powstania}
Algorytm A* powstał w ramach projektu Shakey, zapoczątkowanego w 1966 roku przez Charles Rosen'a.
Celem projektu było zbudowanie robota, który potrafiłby planować własne działania. 
Zbudowany robot wyróżniał się na tle innych tym że intergował kilka różnych modeli sztuczenj 
inteligencji pracujących jako jeden system.

Robot był zbudowany z:
- kamery telewizyjnej i dalmierza optycznego - system wizyjny do obserwacji środowiska
- łącze radiowe - służącego do komunikacji z bazą, odbierania i wysyłania komend
- detektor uderzeń - pozwalający na zatrzymanie robota w przypadku kolizji

Komunikacja odbywała się poprzez wysyłane radiowo tekstowe polecenia mające określoną strukturę np.:
GOTO D4 - co oznaczało automatyczne przemieszczenie się robota do wskazanej pozycji
\begin{figure}[H]
	\centering
	\includegraphics[width=14cm]{pages/algorytm/zdjecia/shakey2.jpg}
	\caption{Robot Shakey i Charles Rosen, inicjator projektu \cite{robotShakey}}
	\label{fig:Rys}
\end{figure}

\subsection{Opis działania algorytmu}

A* to heurestyczny algorytm wyznaczający najkrótszą możliwą ścieżkę w grafie. 
Jest to algorytm zupełny i optymalny, a więc zawsze zostanie wyznaczone optymalne 
rozwiązanie. Ze względu na przeszukiwanie oparte na grafie algorytm działa najlepiej na strukturze drzewiastej.
Zadaniem algorytmu jest minimalizacja funkcji:
\begin{equation}
	f(x)=h(x) + g(x)
	\label{Eq:funkcjaKosztuAStar}
\end{equation}
gdzie: $f(x)$ - minimalizowana funkcja, $g(x)$ - to rzeczywisty koszt dojścia do punktu x.

Funkcja $h(x)$ to funkcja heurestyczna, oszacowuje ona koszt dotarcia od punktu x do wierzchołka docelowego

Zalety:
\begin{itemize}
	\item jest kompletny i optymalny
	\item może przeszukiwać skomplikowane mapy
	\item jest najwydajniejszym takim algorytmem
\end{itemize}
Wady:
\begin{itemize}
	\item jego wydajność w znacznej mierze zależy od funkcji heurestycznej
	\item Każda akcja ma stały koszt wykonania
	\item nie nadaje się do często zmieniającego się otoczenia robota, wymaga ponownego przeliczenia
\end{itemize}


\textbf{Przykładowe funkcje heurestyczne:}
\begin{itemize}
	\item Funkcja euklidesowa
	      \begin{equation}
	      	h(x) = K * \sqrt[2]{(x_1 - x_2)^2 + (y_1 - y_2)^2}
	      	\label{Eq:heuresticEucalides}
	      \end{equation}
	\item Geometria Manhattanu (inaczej metryka miejska)
	      \begin{equation}
	      	h(x) = |x_2 - x_1| + |y_2 - y_1|
	      	\label{Eq:heuresticManhattanu}
	      \end{equation}
	\item Odległość oktylowa
		\begin{equation}
			h(x) = D1 * (|x_1 - x_2| + |y_1 - y_2|) + (D2 - 2*D1) * min(|x_1 - x_2|, |y_1 - y_2|)
			\label{Eq:heuresticOctileDistance}
		\end{equation}

\end{itemize}
Gdzie: $x_1$ i $y_1$ to współrzędne wyznaczanego punktu, $x_2$ i $y_2$ to koordynaty celu, $K$ - oznacza wzmocnienie, $D1$ - koszt skoku na wprost, $D2$ - koszt skoku po przekątnej

\subsection{Analiza innych algorytmów}
\begin{itemize}
	\item{ \textbf{RRT(Rapidly-exploring random tree)}\cite{RRTLec} - to algorytm generujący i łączący 
	losowe punkty w przestrzeni. Z każdym wygenerowanym wierzchołkiem sprawdzane jest czy ten omija przeszkody.
	Jego działanie kończy się gdy węzeł jest wygenerowany we wskazanym regionie lub zostanie osiągnięty limit.
	Jego zalety to:
		\begin{itemize}
			\item Balans pomiędzy zachłannością a eksploracją
			\item prosty i szybki w implementacji
			\item Zbiega się z rozkładem próbkowania
		\end{itemize}
	Jego wady:
		\begin{itemize}
			\item jest czuły na metrykę
			\item duża złożoność obliczeniowa
		\end{itemize}
	}g
	\item{\textbf{SLAM(Simultaneous localization and mapping)} \cite{SLAMMat} - jest metodą używaną w autonomicznych 
	pojazdach. Pozwana na zbudowanie oraz lokalizacje pojazdu względem otoczenia na mapie. Algorytm ten nie wyznacza bezpośrednio
	ścieżki omijającej przeszkody jednak w praktyce jest ważnym elementem takiego systemu. Każdy ruch robota w rzeczywistym 
	środowisku jak i sama mapa obarczona jest pewnym błędem powodującym że robot jest w innym miejscu niż zakładamy. 
	Możemy wyróżnić działanie algorytmu w oparciu o lidary albo o metody wizyjne. Wadą metody jest wysoki koszt obliczeniowy 
	i narastający z czasem błąd pozycji co w konsekwencji może spowodować utratę pozycji.
	\begin{figure}[H]
		\centering
		\includegraphics[width=10cm]{pages/algorytm/zdjecia/slam.jpg}
		\caption{Przykład mapowania metodą SLAM\cite{SLAMMat}}
	\end{figure}
	}
\end{itemize}

