Mechanizm wyszukiwania najkrótszej ścieżki został zamknięty w jednym module o nazwie aStar.
Na moduł składa się klasa AStar ze wszystkimi potrzebnymi metodami oraz klasa Node reprezentująca 
pojedynczy punkt przeszukiwanego grafu. 

Wyznaczanie najkrótszej ścieżki rozpoczyna się od wyznaczenie kosztu punktu startowego, utworzenie zbioru z nieprzeszukanymi 
wierzchołkami, do którego dopisujemy punkt początkowy. 
\begin{lstlisting}[language=Python,caption=Przygotowanie danych,label={kodPython}]
    openList = []
    startNode.h = self.heurestic(startNode.getCords(), endNode.getCords())
    startNode.g = 0
    heappush(openList, startNode)
\end{lstlisting}

Wewnętrzna funkcja heurestic przyjmująca pozycje dwóch punktów odpowiada
za wyliczenie optymistycznego kosztu przejścia od punktu x do wierzchołka docelowego.
Takie podejście pozwala na szybką podmianę funkcji bez znaczących zmian w programie.
Wykorzystywana funkcja heurestyczna to równanie \eqref{Eq:heuresticEucalides} 

W kolejnym kroku uruchamiana jest pętla, która wykonywana jest dopóki
w zbiorze otwartym znajdują się nie odwiedzone elementy. Z listy pobierany jest 
element o najmniejszej liczbie punktów co oznacza że dany wierzchołek drzewa ma największe szanse być najlepszym rozwiązaniem.
Jeżeli pobrany element nie jest celem to dalej pobierani są wszyscy jego sąsiedzi. 
Dalej w pętla przechodzi po pobranych sąsiadach aktualnego punktu. W tym rozwiązaniu dla zaoszczędzenia pamięci, symulator przechowuję tylko przeszkody.
W związku z tym do pobrania sąsiadów używana jest specjalna funkcja sprawdzająca czy na mapie o wskazanych koordynatach istnieje punkt. 
Jeżeli takowy obiekt nie istnieje to oznacza że algorytm może użyć tych współrzędnych do trasowania ścieżki. 
Od tego kodu zależy czy algorytm ma wyznaczyć ścieżkę uwzględniając skoki po skosie.

\begin{lstlisting}[language=Python,caption=Wyznaczenie kosztu ścieżki,label={kodPython2}]
neighborNode = self._findNodeOnList(openList, newCords)
if neighborNode is None:
    neighborNode = Node(newCords)
    neighborNode.g = COST
    neighborNode.h = self.heurestic(newCords, endNode.getCords())
    neighborNode.parent = currentNode
    openList.append(neighborNode)

distance_from_curr_to_neighbor = COST
scoreFromStartToCurrentNeighbor =  currentNode.g + distance_from_curr_to_neighbor

if neighborNode.getScore() <= currentNode.getScore():
    neighborNode.g = scoreFromStartToCurrentNeighbor
    neighborNode.h = scoreFromStartToCurrentNeighbor + self.heurestic(newCords, endNode.getCords())     
    neighborNode.parent = currentNode
\end{lstlisting}

Powyższy kod odpowiada za wyznaczenie kosztu przejścia do sąsiada. Jeżeli punkt nie istnieje jeszcze w otwartym zbiorze 
to jest tworzony i do niego dodawany. Zgodnie z założeniem całkowity koszt to suma rzeczywistego kosztu dystansu i wyniku funkcji heurestycznej. 
Rzeczywisty koszt (funkcja $g(x)$) jest ustalony na sztywno i zależy od współrzędnych, do których skaczemy. 
Jeżeli następny punkt jest na wprost to koszt wynosi 10.
Jako iż przechodzimy po siatce z węzłami o stałej i równej odległości to koszt skoku do sąsiada po skosie został wyznaczony z twierdzenia Pitagorasa. 
Należy zwrócić uwagę że jeżeli całkowity koszt jest mniejszy od aktualnego to węzeł ma ustawianego rodzica, ma to duże znaczenie w przypadku wyjścia z pętli 
i wyznaczenia najkrótszej trasy spośród wszystkich kosztów. 

\begin{lstlisting}[language=Python,caption=Przygotowanie danych,label={kodPython3}]
if currentNode == endNode:
    path = []
    while currentNode is not None:
        path.append(currentNode)
        currentNode = currentNode.getParent()
    return path
else:
    return []
\end{lstlisting}

Po zakończeniu wykonywania pętli, sprawdzane jest czy została znaleziona ścieżka. Jeżeli takowa istnieje to w kolejnej pętli 
wyznaczane są przy pomocy pola wskazującego na rodzica kolejne punkty trasy. 
Jeżeli węzeł nie jest punktem końcowym to trasa nie została znaleziona i zwracana jest pusta lista.
