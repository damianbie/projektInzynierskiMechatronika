\section{Implementacja}
\subsection{Implementacja algorytmu}
Mechanizm wyszukiwania najkrótszej ścieżki został zamknięty w jednym module o nazwie aStar.
Na moduł składa się klasa AStar ze wszystkimi potrzebnymi metodami oraz klasa Node reprezentująca 
pojedynczy punkt przeszukiwanego grafu. 

Wyznaczanie najkrótszej ścieżki rozpoczyna się od wyznaczenie kosztu punktu startowego, utworzenie zbioru z nieprzeszukanymi 
wierzchołkami, do którego dopisujemy punkt początkowy. 
\begin{lstlisting}[language=Python,caption=Przygotowanie danych,label={kodPython}]
    openList = []
    startNode.h = self.heurestic(startNode.getCords(), endNode.getCords())
    startNode.g = 0
    heappush(openList, startNode)
\end{lstlisting}

Wewnętrzna funkcja heurestic przyjmująca pozycje dwóch punktów odpowiada
za wyliczenie optymistycznego kosztu przejścia od punktu x do wierzchołka docelowego.
Takie podejście pozwala na szybką podmianę funkcji bez znaczących zmian w programie.
Wykorzystywana funkcja heurestyczna to równanie \eqref{Eq:heuresticEucalides} 

W kolejnym kroku uruchamiana jest pętla, która wykonywana jest dopóki
w zbiorze otwartym znajdują się nie odwiedzone elementy. Z listy pobierany jest 
element o najmniejszej liczbie punktów co oznacza że dany wierzchołek drzewa jest bliżej od pozostałych.
Jeżeli pobrany element nie jest celem to pobierani są wszyscy jego sąsiedzi.

\subsection{Implementacja środowiska testowego}